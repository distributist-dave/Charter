\documentclass[]{article}
\usepackage{color}

%opening
\title{Canis Society Charter}


\begin{document}

\maketitle


\section{Society Description}
\label{sec:description}

The Canis Society is a non-profit organization for the promotion of justice and Christian community. It exists to promote human thriving in the modern world and the continuance of traditional Western cultures. As such, our core mission is to foster community dedicated to discussion, education, and charity. The Canis Society is a peer-based organization that operates both online and in real life as needed to achieve its objectives. 

\clearpage

\section{Society Mission}
\label{sec:values_mission}

The society expresses several core values along with a set number of long term objectives to guide its administration in the long term. Leadership and members are expected to adhere to these values in their conducts and decisions inside of the society. Long terms progress towards the societies objectives and adherence to its core principles should regularly be revisited to track progress of the leadership. 

\subsection{Society Values}
\label{subsec:values}

The society currently recognizes six core values.

\begin{enumerate}
\item \textbf{\textit{Loyalty}} – The society works for the promotion of human well-being in general, and the well-being of Western peoples in particular. Members as well as leadership are expected to embody loyalty to these causes inside all conduct.
\item \textbf{\textit{Non-violence}} – The society works for peaceful positive change. We condemn  hatred of any religious or ethnic group. While the Society recognizes the necessity of honest discussions concerning religious principles, conversations on privilege or history, and the assertion of ethnic pride or the desire for ethnic autonomy, these critical conversations occur within a context free from defamation or violence.
\item \textbf{\textit{Counter-Progressivism}} – The Canis Society describes itself as explicitly counter-progressive. As such, the Society stands for communities and the traditions that support local communities and against political and corporate campaigns designed to de-construct or erase them.
\item \textbf{\textit{Justice}} – The society stands for fairness in law and the economy for all individuals and communities. The Society values human life, spiritual fulfillment, and the continuation of community between generations.
\item \textbf{\textit{Solidarity}} – The society stands together with other organizations fighting for authentic human relationships and against the destruction of human social organizations by external forces. While other ideological commitments may not allow the society to ally itself with all such political projects, society members and leadership must put the society's core objectives over political and personal rivalries with other organizations that seek similar ends.
\item \textbf{\textit{Leadership}} – The society promotes leadership and living by the principles of community, authenticity, and religiosity. Although no humans are perfect and we do not expect members to be sinless, we encourage those in the organization to strive for these goals both online and off.
\end{enumerate}


\vspace{2mm}
\noindent 
To pursue its values and purpose, we outline here several core values for the society. All subsequent Canis ledership decisions are expected to be conducted to further these core objectives.

\subsection{Society Objectives}
\label{subsec:objectives} 
 
Generally the societies direction is to achieve the following general objectives: 
 
\begin{enumerate}
\item \textbf{\textit{Direct Action}} towards political and charitable ends. Direct Action includes direct giving from the society's treasury, as well as direct volunteering that can be organized locally. This also includes petitioning and political action as becomes necessary. 
\item \textbf{\textit{Community}} building through discussion, recreation, and leisure both online and off. 
\item \textit{\textbf{Spiritual Development}} through prayer and spiritual growth. Although the Canis society is comprised of different faith traditions, the safe development of member spirituality and psychology is a core objective.
\item \textit{\textbf{Education}} through the cultivation of skills and knowledge both practical and theological.
\end{enumerate}

\vspace{2mm}
\noindent 
To obtain these objectives, the President is required to put together a project list for short term achievable goals through which to direct society resources (see Section \ref{sec:leadership}).

\clearpage

\section{Society Ownership and Membership}
\label{sec:members_owners}



\subsection{Membership Responsibilities and Benefits}
\label{subsec:members}

% the basic explnation of the society
Membership in the Canis Society is open to all individuals who are interested in supporting our mission and who adhere to our principles. Members are required to leave contact information, but are not required to be non-anonymous. However, membership may be suspended due to inactivity. 

% the ideological requirements
\vspace{2mm}
\noindent
Members are required to meet ideological requirements explicitly, namely:


% memeber responsability 
\begin{enumerate}
	\item must assert the society's core values and wish to fulfill the objectives of society 
	\item must NOT profess a counter-Western or counter-Christian ideology
	\item sign up for e-mail newsletters and updates (contact requirement).
\end{enumerate}

\vspace{1mm}
\noindent
In exchange members are allowed following rights, including:
% member rights 


\begin{enumerate}
	\item the right to attend and speak at meetings
	\item the right to use the society's images names in blogging and social media 
	\item member voting rights during leadership elections
	\item the right to stand for office in the society 
	\item the right to have access to records of director meetingS (if and when they occur)
\end{enumerate}

\vspace{1mm}
\noindent 
Members are also encouraged to attend society meetings whenever available. 

\subsection{Directorship Responsibilities and Benefits}
\label{subsec:directors}

The number of society directors is intended to include only a limited number of people (for tax purposes). New directors shares will be donated from existing directorship. Furthermore the society requires further restrictions on admitting ANY new directors to admitted. 

\noindent
\vspace{2mm}
Directors are allowed the following privileges:

\begin{enumerate}
\item Constant percentage of society shares
\item Direct ownership (proportionally) of societies assets
\item Right to speak at director meetings
\end{enumerate}


\noindent 
At its inception the society has a single director, its founder David Donovan. 

\clearpage

\section{Society Leadership}
\label{sec:leadership}

% brief run down of executive roles
The society recognizes several \textit{executive} officers which directly manage specific assets inside of the society. Any society member can stand for office. 

\begin{enumerate}
	\item \textbf{\textit{President}}- is the chief executive officer of the society and co-owns all of the other officers' responsibility, he directly owns the society project list and meeting calendar. The President also has the direct responsibility to provide the public face of the Canis society in all communications with external media. The president is required to schedule regular one on ones with the director(s).
	\item \textbf{\textit{Secretary}} - is the executive in charge of communication and updating members via e-mail and social media. The secretary is responsible for publishing the monthly e-mail update and managing the society e-mail list.
	\item \textbf{\textit{Chaplin}} - is the executive in charge of the prayer and spiritual development of the society. The Chaplin directly owns the scheduling of prayers. 
	\item \textbf{\textit{Treasurer}} - is the executive responsible for controlling society funds, keeping the charity list, and maintaining receipts for tax purposes
	\item \textbf{\textit{Technology Administrator}} - is the executive responsible for updating the website and coordinating social media.
	\item \textbf{\textit{Chapter Leader (pending)} }- is any member who starts a chapter with requirements listed in Section \ref{sec:chapter_org}
\end{enumerate}

\noindent
Further instruction on leadership roles will be discussed in future sections. 

\clearpage

\section{Meeting Organization}
\label{sec:meeting_org}

\subsection{General Meeting}
\label{subsec:general_meet}

The society will hold monthly meetings (currently set to be Tuesdays at 6 pm PST and 9 pm EST). Meetings will follow the form as laid out in terms of schedule. 

\begin{enumerate}
	\item deliver opening prayer or solemn secular toast by president or other nominated individual 
	\item review past opens and project list 
	\item each officer reports on area specific issues/concerns
	\item review future concerns and issues with solutions
	\item open floor for discussion (topicality concerns at the presidents discretion)
	\item set time by which all calendars and project lists are to be updated with new decisions. 
	\item parting remarks to society
	\item meeting minutes entered into record
\end{enumerate}

\vspace{2mm}
\noindent 
Update to general meeting form to be discussed as constitution finalized. 


\subsection{Leadership Election}
\label{sec:leader_meet}


\vspace{2mm}
\noindent 
Leadership elections are held held every 12 months or anytime as requested by vote of the directors. During leadership elections all leaders stand for election through the following process. 

\begin{enumerate}
	\item Collect list of candidates for the office (through nomination and acceptance)
	\item Track All Member votes (24 hour period)
	\item Member Votes are weighed with directors not abstaining 
	\item officers selected and given privileges in system
\end{enumerate}

\vspace{2mm}
\noindent 
Outgoing officers are expected to give a brief pass-down in order to bring new leadership up to date on core issues and concerns going forward.

\clearpage

\color{red}

\section{Chapter Organization}
\label{sec:chapter_org}

To be described in future updates to the society constitution. 

\clearpage

\section{Charity Organization}
\label{sec:charity_org}

To be described in future updates to the society constitution. 

\color{black}

\end{document}